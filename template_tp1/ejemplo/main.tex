\documentclass[11pt,a4paper]{article}
\usepackage[utf8]{inputenc}
\usepackage{graphicx}
\usepackage[left=2.5cm,top=3cm,right=2.5cm,bottom=3cm,bindingoffset=0.5cm]{geometry}
\usepackage{AEDLogica, AEDEspecificacion, AEDTADs}
\usepackage{caratula}


\titulo{Trabajo práctico}
\subtitulo{El subtitulo adecuado}

\fecha{\today}

\materia{Algoritmos y Estructuras de Datos}
\grupo{Nombre del grupo} 

\integrante{Cosme, Fulanito}{1113/22}{cosme@algunmail.com}
% \integrante{Apellido, Nombre2}{002/01}{email2@dominio.com} %


% Declaramos donde van a estar las figuras
% No es obligatorio, pero suele ser comodo
\graphicspath{{../static/}}

% Asi pueden escribir nuevos comandos. 
% Este por ejemplo asegura q los nombres 
% que figuren con una tipografia diferenciada  
\newcommand{\Tipo}[1]{\mathsf{#1}}
% la sintaxis es \newcommand{\nombreDeLaMacro}[cantidadDeParametros]{Lo que va ser remplazado por el macro} 
\newcommand{\norm}[1]{\vert #1\vert}




\begin{document}

\maketitle

\section{Macros para escribir en LaTeX}

Para facilitar la escritura del trabajo práctico en LaTeX, les proveemos de varias macros que les permitirán escribir especificaciones en lógica de primer orden.
Por ejemplo, podrán hacer uso de comandos que definen los conectores lógicos: $\land, \lor, \implica$ y también los operadores de la lógica trivaluada $\yLuego$, $\oLuego$, $\implicaLuego$.

Además, disponen de comandos para cuantificadores: $\paraTodo{x}{\mathbb{Z}}{x \geq 0 \lor x < 0}$, $\existe{x}{\mathbb{Z}}{x \geq 0}$. A estos comandos se les puede agregar un \texttt{Largo} al final para que ocupen múltiples renglones.

\section{Ejemplos de uso de las macros}


\begin{tad}{EjemploDeTAD}


\obs{secuenciaDeCaracteres}{\seq{\cha}}
\\
\obs{conjuntoDeTuplas}{\conj{\tupla{nombre: \seq{\cha}, apellido: \seq{\cha}}}}
\\


\begin{proc}{crearTAD}
{
\In caracteres: \seq{\cha},
\In secuenciaDeNombres: \seq{\tupla{nombre: \seq{\cha}, apellido: \seq{\cha}}}
}{
\Tipo{EjemploDeTAD}
}
    \requiere{predicadoUnilinea(caracteres)}
    \asegura{res.secuenciaDeCaracteres = caracteres}
    \aseguraLargo{ \norm{res.conjuntoDeTuplas} = \norm{secuenciaDeNombres} \land \\predicadoMultilinea(secuenciaDeNombres, res.conjuntoDeTuplas) }
\end{proc}

\predLargo{predicadoMultilinea}{s: \seq{\cha}, conjunto: \conj{\tupla{\seq{\cha}, \seq{\cha}}}}{
    \paraTodo{i}{\Z}{0 \leq i < \norm{s} 
    \implicaLuego s[i] \in c}
}

\pred{predicadoUnilínea}{s: \seq{\cha}}{
    \norm{s} > 0
}

\aux{unAuxiliar}{k: \Z,l: \Z}{\Z}{\norm{k-l}}

\end{tad}




\end{document}
